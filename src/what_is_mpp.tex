\documentclass{article}
\title{MPP: a meta preprocessor and a language blender}
\author{Philippe Wang}

\begin{document}
\maketitle
MPP is yet another preprocessor.  As some other few preprocessors, MPP
is generic  and may be  used with any text-based  programming language
(e.g.,  ASM, C,  Java, ML,  Scheme) or  document  description language
(e.g., HTML, LaTeX, Markdown). MPP may be given parameters in order to
restrict or loosen its usage.

MPP  is  a tool  designed  to  be used  both  by  non programmers  and
programmers. First, it means that it is easy to use. For that, we made
the syntax very simple and there is no need to understand for instance
what a  function is. Second, it  means that programmers  should not be
frustrated when using it. For  that, by design, MPP allows programmers
to use any language as a preprocessor.

For programmers who  are familiar with OCaml, it  is important to know
that  MPP  is  designed  to   be  easily  extensible.  It  has  a  few
entry-points that are designed for new features to be plugged.

MPP will  stop on undefined variables,  which is a  feature that helps
finding errors.   MPP allows to easily blend  programming languages in
order to use them as  preprocessor languages. It is of course possible
to use tools  such as Bash or Perl to  blend languages.  The advantage
of MPP over such tools is that MPP is specially designed for that.

MPP has a few layers.

- MPP blocks: execute an action.
- MPP special blocks: convert everything into a program that will output a processed version.
- Not MPP blocks: verbatim output.

\end{document}
