% -*- coding: utf-8; -*-
\documentclass[a4paper]{article}

\usepackage{geometry} \geometry{reset, papersize={210.00mm,296.9mm},
  left=25mm, right=25mm, vmargin=20mm, text={160mm,220mm},
  includeheadfoot, headheight=0em, headsep=4em, hoffset=0cm,
  voffset=0cm, footskip=4.5em, }

\usepackage{listings}
\usepackage{xcolor}
\lstset{
  basicstyle=\ttfamily,
  backgroundcolor=\color{black!10},
}

\title{MPP: a meta preprocessor and a language blender}


\author{Philippe Wang\\\texttt{Philippe.Wang@cl.cam.ac.uk}}
\date{June 7, 2013}

\def\eg{\emph{e.g.}}

\begin{document}
\maketitle
\begin{center}
  OCaml Labs\\
  The University of Cambridge Computer Laboratory, UK\\
\end{center}
\section{MPP is yet another preprocessor}
As some other few preprocessors, MPP
is  designed  to  be  generic  and  to be  used  with  any  text-based
programming  language (e.g.,  ASM, C,  Java, ML,  Scheme)  or document
description language  (e.g., HTML, LaTeX, Markdown). MPP  may be given
parameters in  order to restrict or  loosen its usage. It  can also be
extended to provide additional features.

\section{MPP is a tool designed to be easy to use}

The syntax of MPP was made very simple and there is very little need
to understand what ``programming'' means.  Moreover, it is possible to
easily customize the syntax of MPP, so that MPP can provide precisely
what its users need.

Programmers should not be frustrated when using MPP. For that, MPP can
be easily  extended and be  used to generate  a customized/specialized
preprocessor.

MPP takes a file and processes its contents.  The output is the output
from interpretation of special commands  and verbatim copy the rest of
the  file.  The  usage may  be customized,  by using  special built-in
commands,  command line  options, or  by modifying  the implementation
(which is designed to be easy to modify).

Here  follows an  example  of a  LaTeX  file that  is  intended to  be
processed by MPP, where tokens \texttt{((} and \texttt{))} are used to
delimit a special command:
\begin{lstlisting}
Executing the following command \texttt{mkdir -p a/b/c} 
% (( cmd mkdir -p a/b/c ))
creates a few directories.
Here's the result of running the \texttt{tree} command afterwards:
\begin{verbatim}
(( cmd tree ))
\end{verbatim}
\end{lstlisting}

and the output is:
\begin{lstlisting}
Executing the following command \texttt{mkdir -p a/b/c} 
% 
creates a few directories.
Here's the result of running the \texttt{tree} command afterwards:  
\begin{verbatim}
.
`-- a
    `-- b
        `-- c

3 directories, 0 files
\end{verbatim}
\end{lstlisting}
N.B. This example used the \texttt{cmd} built-in command to execute an external
shell command.

\section{MPP  is a language  blender}

MPP  allows  any  programming  language  to  be  smoothly  used  as  a
preprocessing language.  Note that languages such as Perl or Bash also
allow this kind  of practice however MPP is designed to  make it a lot
easier, safer and more relevant.

Let Tyu be a programming  language or a document description language.
Let L be a programming language.   Say that one is writing a .tyu file
and  wants  to  use  the  L programming  language  as  a  preprocessing
language. 

MPP makes this possible if it has been given a description to allow it
to convert the  .tyu file into a program in the  L language that, when
executed, prints back a .tuy  file. Inside the original .tyu file, one
may  declare special  blocks and  use the  L language,  so  that those
blocks are not converted but simply copied verbatim.
This way, all blocks written in the L language share the same execution
environment.

For instance, if TUY=HTML and L=OCaml:

Here follow the contents of file \texttt{f.html} to be processed by MPP and produces
\texttt{tmp.ml}:
\begin{lstlisting}
<table>
{{
let () = 
for i = 1 to 5 do
 Printf.printf "<tr>";
 for j = 1 to 5 do
  Printf.printf "<td>%d</td>" (i*j)
 done;
 Printf.printf "</tr>\n";
done
}}
</table>
\end{lstlisting}

Here follow the contents of \texttt{tmp.ml}:
\begin{lstlisting}
let () = Printf.printf "<table>\n"
let () = 
for i = 1 to 5 do
 Printf.printf "<tr>";
 for j = 1 to 5 do
  Printf.printf "<td>%d</td>" (i*j)
 done;
 Printf.printf "</tr>\n";
done
let () = Printf.printf "</table>\n"
\end{lstlisting}

The resulting file is:
\begin{lstlisting}
<table>
<tr><td>1</td><td>2</td><td>3</td><td>4</td><td>5</td></tr>
<tr><td>2</td><td>4</td><td>6</td><td>8</td><td>10</td></tr>
<tr><td>3</td><td>6</td><td>9</td><td>12</td><td>15</td></tr>
<tr><td>4</td><td>8</td><td>12</td><td>16</td><td>20</td></tr>
<tr><td>5</td><td>10</td><td>15</td><td>20</td><td>25</td></tr>
</table>
\end{lstlisting}

As one  can see, tokens \verb+{{+  and \verb+}}+ allowed  to declare a
special section  in file \texttt{f.html}.  Note that  these two tokens
may be customized by the user of MPP when calling it.

By invoking MPP multiple times with different block delimiters, one
may blend as many languages as he/she may want to.

%\section{Notes on MPP}
The commands part and the language  blending part of MPP may happen in
the same MPP execution or separately, this choice is up to the user.



\section{Notes about MPP}
\begin{itemize}
\item MPP is implemented in OCaml and its set of features can be
  extended by registering additional features that are simple OCaml
  functions.

\item MPP may also be used as a library for its simple generic
  lexing/parsing engine.


\item MPP is not taking the approach of embedding something into
  OCaml. Instead, it allows to embed the language of your choice into
  any existing document.


\item MPP allows OCaml to be used as a preprocessing language for any
  program or document text file.  It provides a generic mechanism to
  allow the use of other programming languages as preprocessing
  languages.

\item Let us note that, by default, MPP stops if any error occurs
  (\eg, access to an unbound variable, or external program exiting
  with nonzero), but options are provided to ignore such errors.

\item MPP is not intended to replace a tool such as Camlp4 or ppx,
  which are meant to extend the OCaml language. Instead, MPP provides
  a small set of built-in commands, and allows you to use any language
  of your choice as a text preprocessor.

\item MPP is free and open source software.
\end{itemize}

\section{Summary}
MPP is a preprocessor: one can use MPP to preprocess any text file.

MPP is a meta preprocessor: one can use MPP to generate a customized preprocessor.

MPP is a language blender: one can use MPP to make virtually any programming language 
their preprocessor language. 

MPP is also a set of OCaml modules that may be used as a library.

MPP provides simple interfaces for programmers to extend MPP, using OCaml functions.

MPP  is an  original and  powerful tool  for programmers but is also
designed to be useful for non-programmers, who may use the core of MPP
or a specialized version of MPP.


\end{document}
